\section{Preliminaries}
給定兩個電路Cir1與Cir2,他們的PIs與POs分別為X,Y與F,G,代表PI或PO的集合,並且寫作$X = \{x_1, x_2, ... , x_n\}$, $Y = \{y_1, y_2, ... , y_m\}$, $F = \{f_1, f_2, ... , f_k\}$, $G = \{g_1, g_2, ... , x_l\}$,另外使用$MUX(A, c_1, c_2, ... ,c_t)$來表示以A集合中的變數來選取c_1至c_t 的其中一個,將其中一個值傳至輸出端。

\subsection{NP3 Boolean Matching}
給定兩個個電路Cir1與Cir2,對於電路Cir2的PI集合Y的元素,$y_i \in \{0, 1, x_1, \overline{x_1}, x_2, \overline{x_2}, ... , x_n, \overline{x_n}\}$,在此條件下,對於Cir2的PO集合G的元素,$g_i \in \{f_1, \overline{f_1}, f_2, \overline{f_2}, ... , f_n, \overline{f_n}, other\}$,目標是找到一組對應關係,使得給定任意的inputs,有最多對的$g_i \equiv f_j$。

\subsection{Quantified Boolean Formula}
我們可以將NP3 Boolean Matching 的問題轉成一個QBF的問題,而QBF是廣義版的Boolean satisfiability problem,以另一個說法,QBF詢問對於一連串Quantified Boolean varibles with order,formula \phi 是否有解,以下舉一個例子:

\[ \forall x\ \exists y\ \exists z\ .((x\lor z)\land y) \]

在一般的狀況下,我們會將式子寫成這樣的形式:

\[ \exists X_1\ \forall X_2\ \cdots Q_nX_n .~\phi \]

每一個$Q \in \{\forall, \exist\}$代表一種 Quantifier,每一個$X_i$代表一個variables block,由相鄰的一樣 Quantifier 的變數所組成。解決QBF這樣形式的問題,近年來有非常多研究[6,7,8],因此如果將問題轉變成QBF,便能省去許多證明流程上的複雜性,但相對來說,會使得解決的效率變差。

