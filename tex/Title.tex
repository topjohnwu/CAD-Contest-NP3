% paper title
% Titles are generally capitalized except for words such as a, an, and, as,
% at, but, by, for, in, nor, of, on, or, the, to and up, which are usually
% not capitalized unless they are the first or last word of the title.
% Linebreaks \\ can be used within to get better formatting as desired.
% Do not put math or special symbols in the title.
\title{105-2~~EDA Final Project \\ NP3: Non-exact Projective NPNP Boolean Matching}


% author names and affiliations
% use a multiple column layout for up to three different
% affiliations
\author{\IEEEauthorblockN{Yen-Shi Wang}
\IEEEauthorblockA{
EE Dept. \\
National Taiwan University \\
b03901116@ntu.edu.tw}
\and
\IEEEauthorblockN{John Wu}
\IEEEauthorblockA{
EE Dept. \\
National Taiwan University \\
b03901034@ntu.edu.tw}
\and}

% conference papers do not typically use \thanks and this command
% is locked out in conference mode. If really needed, such as for
% the acknowledgment of grants, issue a \IEEEoverridecommandlockouts
% after \documentclass

% for over three affiliations, or if they all won't fit within the width
% of the page (and note that there is less available width in this regard for
% compsoc conferences compared to traditional conferences), use this
% alternative format:
% 
%\author{\IEEEauthorblockN{Michael Shell\IEEEauthorrefmark{1},
%Homer Simpson\IEEEauthorrefmark{2},
%James Kirk\IEEEauthorrefmark{3}, 
%Montgomery Scott\IEEEauthorrefmark{3} and
%Eldon Tyrell\IEEEauthorrefmark{4}}
%\IEEEauthorblockA{\IEEEauthorrefmark{1}School of Electrical and Computer Engineering\\
%Georgia Institute of Technology,
%Atlanta, Georgia 30332--0250\\ Email: see http://www.michaelshell.org/contact.html}
%\IEEEauthorblockA{\IEEEauthorrefmark{2}Twentieth Century Fox, Springfield, USA\\
%Email: homer@thesimpsons.com}
%\IEEEauthorblockA{\IEEEauthorrefmark{3}Starfleet Academy, San Francisco, California 96678-2391\\
%Telephone: (800) 555--1212, Fax: (888) 555--1212}
%\IEEEauthorblockA{\IEEEauthorrefmark{4}Tyrell Inc., 123 Replicant Street, Los Angeles, California 90210--4321}}

% use for special paper notices
%\IEEEspecialpapernotice{(Invited Paper)}

% make the title area
\maketitle

% As a general rule, do not put math, special symbols or citations
% in the abstract
\begin{abstract}
NP3 Boolean Matching Problem is a problem to find the maximum functional equivalence output groups by negating or permuting the PI and PO of two circuits. As this problem extends to NPNP and PP equivalence matching, it can be used further in issues like design rectification, technology mapping, and logic synthesis application.

In this research project, we implemented our solution on top of the tool ABC, using it's AIG and QBF functions along with ILP circuits to optimize the result.
% 的問題,是給定兩個電路,目標是尋找:在PI與PO可以經由排列與反向後,可以對應到function上完全一樣的最多對數。在design rectification, technology mapping, 以及 logic synthesis application中都會使用到,由於此問題更加廣義化了一般NPNP或PP equivalence matching 的問題,所以更加值得研究。在這一份研究報告中,我們在ABC上實做了一個使用AIG並與QBF作結合的Solver,並使用ILP電路來使成效最佳化。

\end{abstract}

% no keywords

% For peer review papers, you can put extra information on the cover
% page as needed:
% \ifCLASSOPTIONpeerreview
% \begin{center} \bfseries EDICS Category: 3-BBND \end{center}
% \fi
%
% For peerreview papers, this IEEEtran command inserts a page break and
% creates the second title. It will be ignored for other modes.
\IEEEpeerreviewmaketitle
