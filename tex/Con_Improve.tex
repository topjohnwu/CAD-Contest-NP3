\section{Conclusion}
我們使用ABC system中建構的AIG模組,讀取兩個verilog的電路檔,並連接MUX以及ILP電路,最後使用裡面內建的QBF solver來解決NP3 Boolean Matching的問題,在效能上有許多問題並沒有考量到,由於MUX個數過多,以及QBF solver的效能不太能控制,也沒有使用外部的solver,因此我們所建構的NP3 solver算是一個簡單完整的程式,只是並不能保證效能。

\section{Improvement}
如果要使效能變快,一定不能整個電路都扔進QBF solver中,這樣的search space實在太過龐大,考量到這個部份,接下來會朝幾個方向來思考如何進步。

第一點是觀察兩個電路中每個PO的structural supports與functional supports數量,可以利用這兩個值作為上下界,並再擴張"兩個電路PI數量的差距",只要兩個PO的範圍沒有重疊則表示這兩個PO不可能functional equivalence,可以由此建構出每個Cir1的PO可能對應到Cir2的PO的表,並根據此表的對應關係來作MUX或是進一步的Combinational Equivalence Checking。

再來則是對於兩個已知有機會相等的PO,如果先不考慮PI端的negation,可以利用[1]中所提到的方法,經由simulation來將Cir1與Cir2的PI分別作分堆的動作,並且依據分出堆的先後順序與大小排列,即可根據兩邊的分堆來作PI的對應,可以大量減少可能對應的對數。

另一個方式則是在做完supports的觀察後,決定先從PO的fanin con最小的開始作配對,直接扔進QBF solver,看是否有解,如果有解則將PI對應的順序記錄下來,以此為條件繼續作fanin con 數量第二小的PO的配對,持續直到conflict產生,可以跳過這組PO配對或者重頭來過,並以這樣的算法持續運作。

\section{Jod Division}
John Wu: 連接QBF solver、測試自動化與輸出解答檔案、撰寫PPT內容、Report修訂與改良\\

Yen-Shi Wang: 於ABC中建構整個電路、製作PPT、Report初版\\

% conference papers do not normally have an appendix
