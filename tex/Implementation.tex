\section{Implementation}
In this section we will explain how we construct this simple solver in detail. First we will talk about the tool we use - ABC. We will briefly introduce the internal data structure and the API we used. Then we will show the circuit structure in AIG. Finally, we will illustrate the usage of ILP in the problem, the method to iteratively optimize the solution, and how to construct circuits to represent in-equations such as $a+b+c \geq 2$.
% 在這個段落會詳細講解我們如何建構這個簡單的Solver,首先介紹我們所使用的ABC package,裡面內建的部份資料結構以及可以使用的API;接下來因為我們會使用AIG的方式來儲存整個電路,所以會展示電路的架構圖;最後一部分使用ILP來達到iterative遞增解決問題的方式,簡單說明如何以電路設計類似$a+b+c \geq 2$的不等式。
\subsection{ABC Package}
%<<<<<<< 920e56612cbbe2469913aaec0ed7326f74b9631d
ABC is A System for Sequential Synthesis and Verification written by Berkeley Logic Synthesis and Verification Group, it combines scalable logic optimization based on And-Inverter Graphs (AIGs), optimal-delay DAG-based technology mapping for look-up tables and standard cells, and innovative algorithms for sequential synthesis and verification.
% 我們主要使用read_verilog這個指令來有效的讀取兩個電路檔,並參考ABC中的miter與[9]來建構整個電路,
% ABC 是 A System for Sequential Synthesis and Verification 由 Berkely的Berkeley Logic Synthesis and Verification Group所建構,他可以用AIG的形式處理可擴張的電路,並且以DAG為基礎,最佳延遲為目標,在系統中運作時都會進行technology mapping。
% 我們主要使用read_verilog這個指令來有效的讀取兩個電路檔,並參考ABC中的miter與[9]來建構整個電路,
%=======
%ABC 是 A System for Sequential Synthesis and Verification 由 Berkely的Berkeley Logic Synthesis and Verification Group所建構,他可以用AIG的形式處理可擴張的電路,並且以DAG為基礎,最佳延遲為目標,在系統中運作時都會進行technology mapping。
%我們主要使用read\_verilog這個指令來有效的讀取兩個電路檔,並參考ABC中的miter與[9]來建構整個電路,
%>>>>>>> rebase
\subsection{AIG Structure}
An and-inverter graph (AIG) is a directed, acyclic graph  that represents a structural implementation of the logical functionality of a circuit or network.\\
每一個AIG中的node代表一個電路圖中的And gate,並且在每一個input端可以決定這個值是否要被翻轉。以這種形式表示電路來處理Boolean satisfiability在formal verification的領域裡有非常大的成效,包含model checking以及equivalence checking。
因此在將兩個電路讀進ABC system後,進行strash的動作,直接將整個電路圖轉變成為AIG的形式以進行接下來的操作,如圖1所示。
% 放ppt的圖
\subsection{ILP Circuit}
在圖中,上半部份是由n個MUX所組成,每一個MUX會從Cir2的outputs裡面選一個作為輸出,輸出之後與Cir1中的每一個output連接至一個XOR gate,將所有的XOR的輸出連接至一個OR gate後,將控制輸出是否為CONST0的MUX連接至ILP circuit,最後將ILP的輸出與OR gate的輸出and在一起,得到最後的OUTPUT。

