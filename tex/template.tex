%% bare_conf_compsoc.tex
%% V1.4b
%% 2015/08/26
%% by Michael Shell
%% See:
%% http://www.michaelshell.org/
%% for current contact information.
%%
%% This is a skeleton file demonstrating the use of IEEEtran.cls
%% (requires IEEEtran.cls version 1.8b or later) with an IEEE Computer
%% Society conference paper.
%%
%% Support sites:
%% http://www.michaelshell.org/tex/ieeetran/
%% http://www.ctan.org/pkg/ieeetran
%% and
%% http://www.ieee.org/

%%*************************************************************************
%% Legal Notice:
%% This code is offered as-is without any warranty either expressed or
%% implied; without even the implied warranty of MERCHANTABILITY or
%% FITNESS FOR A PARTICULAR PURPOSE! 
%% User assumes all risk.
%% In no event shall the IEEE or any contributor to this code be liable for
%% any damages or losses, including, but not limited to, incidental,
%% consequential, or any other damages, resulting from the use or misuse
%% of any information contained here.
%%
%% All comments are the opinions of their respective authors and are not
%% necessarily endorsed by the IEEE.
%%
%% This work is distributed under the LaTeX Project Public License (LPPL)
%% ( http://www.latex-project.org/ ) version 1.3, and may be freely used,
%% distributed and modified. A copy of the LPPL, version 1.3, is included
%% in the base LaTeX documentation of all distributions of LaTeX released
%% 2003/12/01 or later.
%% Retain all contribution notices and credits.
%% ** Modified files should be clearly indicated as such, including  **
%% ** renaming them and changing author support contact information. **
%%*************************************************************************


% *** Authors should verify (and, if needed, correct) their LaTeX system  ***
% *** with the testflow diagnostic prior to trusting their LaTeX platform ***
% *** with production work. The IEEE's font choices and paper sizes can   ***
% *** trigger bugs that do not appear when using other class files.       ***                          ***
% The testflow support page is at:
% http://www.michaelshell.org/tex/testflow/



\documentclass[conference,compsoc]{IEEEtran}
% Some/most Computer Society conferences require the compsoc mode option,
% but others may want the standard conference format.
%
% If IEEEtran.cls has not been installed into the LaTeX system files,
% manually specify the path to it like:
% \documentclass[conference,compsoc]{../sty/IEEEtran}





% Some very useful LaTeX packages include:
% (uncomment the ones you want to load)


% *** MISC UTILITY PACKAGES ***
%
%\usepackage{ifpdf}
% Heiko Oberdiek's ifpdf.sty is very useful if you need conditional
% compilation based on whether the output is pdf or dvi.
% usage:
% \ifpdf
%   % pdf code
% \else
%   % dvi code
% \fi
% The latest version of ifpdf.sty can be obtained from:
% http://www.ctan.org/pkg/ifpdf
% Also, note that IEEEtran.cls V1.7 and later provides a builtin
% \ifCLASSINFOpdf conditional that works the same way.
% When switching from latex to pdflatex and vice-versa, the compiler may
% have to be run twice to clear warning/error messages.






% *** CITATION PACKAGES ***
%
\ifCLASSOPTIONcompsoc
  % IEEE Computer Society needs nocompress option
  % requires cite.sty v4.0 or later (November 2003)
  \usepackage[nocompress]{cite}
\else
  % normal IEEE
  \usepackage{cite}
\fi
% cite.sty was written by Donald Arseneau
% V1.6 and later of IEEEtran pre-defines the format of the cite.sty package
% \cite{} output to follow that of the IEEE. Loading the cite package will
% result in citation numbers being automatically sorted and properly
% "compressed/ranged". e.g., [1], [9], [2], [7], [5], [6] without using
% cite.sty will become [1], [2], [5]--[7], [9] using cite.sty. cite.sty's
% \cite will automatically add leading space, if needed. Use cite.sty's
% noadjust option (cite.sty V3.8 and later) if you want to turn this off
% such as if a citation ever needs to be enclosed in parenthesis.
% cite.sty is already installed on most LaTeX systems. Be sure and use
% version 5.0 (2009-03-20) and later if using hyperref.sty.
% The latest version can be obtained at:
% http://www.ctan.org/pkg/cite
% The documentation is contained in the cite.sty file itself.
%
% Note that some packages require special options to format as the Computer
% Society requires. In particular, Computer Society  papers do not use
% compressed citation ranges as is done in typical IEEE papers
% (e.g., [1]-[4]). Instead, they list every citation separately in order
% (e.g., [1], [2], [3], [4]). To get the latter we need to load the cite
% package with the nocompress option which is supported by cite.sty v4.0
% and later.





% *** GRAPHICS RELATED PACKAGES ***
%
\ifCLASSINFOpdf
  % \usepackage[pdftex]{graphicx}
  % declare the path(s) where your graphic files are
  % \graphicspath{{../pdf/}{../jpeg/}}
  % and their extensions so you won't have to specify these with
  % every instance of \includegraphics
  % \DeclareGraphicsExtensions{.pdf,.jpeg,.png}
\else
  % or other class option (dvipsone, dvipdf, if not using dvips). graphicx
  % will default to the driver specified in the system graphics.cfg if no
  % driver is specified.
  % \usepackage[dvips]{graphicx}
  % declare the path(s) where your graphic files are
  % \graphicspath{{../eps/}}
  % and their extensions so you won't have to specify these with
  % every instance of \includegraphics
  % \DeclareGraphicsExtensions{.eps}
\fi
% graphicx was written by David Carlisle and Sebastian Rahtz. It is
% required if you want graphics, photos, etc. graphicx.sty is already
% installed on most LaTeX systems. The latest version and documentation
% can be obtained at: 
% http://www.ctan.org/pkg/graphicx
% Another good source of documentation is "Using Imported Graphics in
% LaTeX2e" by Keith Reckdahl which can be found at:
% http://www.ctan.org/pkg/epslatex
%
% latex, and pdflatex in dvi mode, support graphics in encapsulated
% postscript (.eps) format. pdflatex in pdf mode supports graphics
% in .pdf, .jpeg, .png and .mps (metapost) formats. Users should ensure
% that all non-photo figures use a vector format (.eps, .pdf, .mps) and
% not a bitmapped formats (.jpeg, .png). The IEEE frowns on bitmapped formats
% which can result in "jaggedy"/blurry rendering of lines and letters as
% well as large increases in file sizes.
%
% You can find documentation about the pdfTeX application at:
% http://www.tug.org/applications/pdftex





% *** MATH PACKAGES ***
%
%\usepackage{amsmath}
% A popular package from the American Mathematical Society that provides
% many useful and powerful commands for dealing with mathematics.
%
% Note that the amsmath package sets \interdisplaylinepenalty to 10000
% thus preventing page breaks from occurring within multiline equations. Use:
%\interdisplaylinepenalty=2500
% after loading amsmath to restore such page breaks as IEEEtran.cls normally
% does. amsmath.sty is already installed on most LaTeX systems. The latest
% version and documentation can be obtained at:
% http://www.ctan.org/pkg/amsmath





% *** SPECIALIZED LIST PACKAGES ***
%
%\usepackage{algorithmic}
% algorithmic.sty was written by Peter Williams and Rogerio Brito.
% This package provides an algorithmic environment fo describing algorithms.
% You can use the algorithmic environment in-text or within a figure
% environment to provide for a floating algorithm. Do NOT use the algorithm
% floating environment provided by algorithm.sty (by the same authors) or
% algorithm2e.sty (by Christophe Fiorio) as the IEEE does not use dedicated
% algorithm float types and packages that provide these will not provide
% correct IEEE style captions. The latest version and documentation of
% algorithmic.sty can be obtained at:
% http://www.ctan.org/pkg/algorithms
% Also of interest may be the (relatively newer and more customizable)
% algorithmicx.sty package by Szasz Janos:
% http://www.ctan.org/pkg/algorithmicx




% *** ALIGNMENT PACKAGES ***
%
%\usepackage{array}
% Frank Mittelbach's and David Carlisle's array.sty patches and improves
% the standard LaTeX2e array and tabular environments to provide better
% appearance and additional user controls. As the default LaTeX2e table
% generation code is lacking to the point of almost being broken with
% respect to the quality of the end results, all users are strongly
% advised to use an enhanced (at the very least that provided by array.sty)
% set of table tools. array.sty is already installed on most systems. The
% latest version and documentation can be obtained at:
% http://www.ctan.org/pkg/array


% IEEEtran contains the IEEEeqnarray family of commands that can be used to
% generate multiline equations as well as matrices, tables, etc., of high
% quality.




% *** SUBFIGURE PACKAGES ***
%\ifCLASSOPTIONcompsoc
%  \usepackage[caption=false,font=footnotesize,labelfont=sf,textfont=sf]{subfig}
%\else
%  \usepackage[caption=false,font=footnotesize]{subfig}
%\fi
% subfig.sty, written by Steven Douglas Cochran, is the modern replacement
% for subfigure.sty, the latter of which is no longer maintained and is
% incompatible with some LaTeX packages including fixltx2e. However,
% subfig.sty requires and automatically loads Axel Sommerfeldt's caption.sty
% which will override IEEEtran.cls' handling of captions and this will result
% in non-IEEE style figure/table captions. To prevent this problem, be sure
% and invoke subfig.sty's "caption=false" package option (available since
% subfig.sty version 1.3, 2005/06/28) as this is will preserve IEEEtran.cls
% handling of captions.
% Note that the Computer Society format requires a sans serif font rather
% than the serif font used in traditional IEEE formatting and thus the need
% to invoke different subfig.sty package options depending on whether
% compsoc mode has been enabled.
%
% The latest version and documentation of subfig.sty can be obtained at:
% http://www.ctan.org/pkg/subfig




% *** FLOAT PACKAGES ***
%
%\usepackage{fixltx2e}
% fixltx2e, the successor to the earlier fix2col.sty, was written by
% Frank Mittelbach and David Carlisle. This package corrects a few problems
% in the LaTeX2e kernel, the most notable of which is that in current
% LaTeX2e releases, the ordering of single and double column floats is not
% guaranteed to be preserved. Thus, an unpatched LaTeX2e can allow a
% single column figure to be placed prior to an earlier double column
% figure.
% Be aware that LaTeX2e kernels dated 2015 and later have fixltx2e.sty's
% corrections already built into the system in which case a warning will
% be issued if an attempt is made to load fixltx2e.sty as it is no longer
% needed.
% The latest version and documentation can be found at:
% http://www.ctan.org/pkg/fixltx2e


%\usepackage{stfloats}
% stfloats.sty was written by Sigitas Tolusis. This package gives LaTeX2e
% the ability to do double column floats at the bottom of the page as well
% as the top. (e.g., "\begin{figure*}[!b]" is not normally possible in
% LaTeX2e). It also provides a command:
%\fnbelowfloat
% to enable the placement of footnotes below bottom floats (the standard
% LaTeX2e kernel puts them above bottom floats). This is an invasive package
% which rewrites many portions of the LaTeX2e float routines. It may not work
% with other packages that modify the LaTeX2e float routines. The latest
% version and documentation can be obtained at:
% http://www.ctan.org/pkg/stfloats
% Do not use the stfloats baselinefloat ability as the IEEE does not allow
% \baselineskip to stretch. Authors submitting work to the IEEE should note
% that the IEEE rarely uses double column equations and that authors should try
% to avoid such use. Do not be tempted to use the cuted.sty or midfloat.sty
% packages (also by Sigitas Tolusis) as the IEEE does not format its papers in
% such ways.
% Do not attempt to use stfloats with fixltx2e as they are incompatible.
% Instead, use Morten Hogholm'a dblfloatfix which combines the features
% of both fixltx2e and stfloats:
%
% \usepackage{dblfloatfix}
% The latest version can be found at:
% http://www.ctan.org/pkg/dblfloatfix




% *** PDF, URL AND HYPERLINK PACKAGES ***
%
%\usepackage{url}
% url.sty was written by Donald Arseneau. It provides better support for
% handling and breaking URLs. url.sty is already installed on most LaTeX
% systems. The latest version and documentation can be obtained at:
% http://www.ctan.org/pkg/url
% Basically, \url{my_url_here}.




% *** Do not adjust lengths that control margins, column widths, etc. ***
% *** Do not use packages that alter fonts (such as pslatex).         ***
% There should be no need to do such things with IEEEtran.cls V1.6 and later.
% (Unless specifically asked to do so by the journal or conference you plan
% to submit to, of course. )

\usepackage[CheckSingle, CJKmath]{xeCJK}  % xelatex 中文
\setCJKmainfont[BoldFont={Droid Sans Fallback},Scale=0.95]{I.BMing}
\newCJKfontfamily\Hei{Droid Sans Fallback}
\usepackage{CJKulem}	% 中文字裝飾

% correct bad hyphenation here
\hyphenation{op-tical net-works semi-conduc-tor}


\begin{document}
% paper title
% Titles are generally capitalized except for words such as a, an, and, as,
% at, but, by, for, in, nor, of, on, or, the, to and up, which are usually
% not capitalized unless they are the first or last word of the title.
% Linebreaks \\ can be used within to get better formatting as desired.
% Do not put math or special symbols in the title.
\title{NP3: Non-exact Projective NPNP Boolean Matching}


% author names and affiliations
% use a multiple column layout for up to three different
% affiliations
\author{\IEEEauthorblockN{Yen-Shi Wang}
\IEEEauthorblockA{
School of Electrical and\\Computer Engineering\\
Georgia Institute of Technology\\
Atlanta, Georgia 30332--0250\\
Email: b03901116@ntu.edu.tw}
\and
\IEEEauthorblockN{John Wu}
\IEEEauthorblockA{Twentieth Century Fox\\
Springfield, USA\\
Email: b03901034@ntu.edu.tw}
\and}

% conference papers do not typically use \thanks and this command
% is locked out in conference mode. If really needed, such as for
% the acknowledgment of grants, issue a \IEEEoverridecommandlockouts
% after \documentclass

% for over three affiliations, or if they all won't fit within the width
% of the page (and note that there is less available width in this regard for
% compsoc conferences compared to traditional conferences), use this
% alternative format:
% 
%\author{\IEEEauthorblockN{Michael Shell\IEEEauthorrefmark{1},
%Homer Simpson\IEEEauthorrefmark{2},
%James Kirk\IEEEauthorrefmark{3}, 
%Montgomery Scott\IEEEauthorrefmark{3} and
%Eldon Tyrell\IEEEauthorrefmark{4}}
%\IEEEauthorblockA{\IEEEauthorrefmark{1}School of Electrical and Computer Engineering\\
%Georgia Institute of Technology,
%Atlanta, Georgia 30332--0250\\ Email: see http://www.michaelshell.org/contact.html}
%\IEEEauthorblockA{\IEEEauthorrefmark{2}Twentieth Century Fox, Springfield, USA\\
%Email: homer@thesimpsons.com}
%\IEEEauthorblockA{\IEEEauthorrefmark{3}Starfleet Academy, San Francisco, California 96678-2391\\
%Telephone: (800) 555--1212, Fax: (888) 555--1212}
%\IEEEauthorblockA{\IEEEauthorrefmark{4}Tyrell Inc., 123 Replicant Street, Los Angeles, California 90210--4321}}




% use for special paper notices
%\IEEEspecialpapernotice{(Invited Paper)}




% make the title area
\maketitle

% As a general rule, do not put math, special symbols or citations
% in the abstract
\begin{abstract}
簡單介紹NP3問題、應用,我們以ABC所建構的Solver,主要結構以及實驗結果、能進步的地方。
\end{abstract}

% no keywords

% For peer review papers, you can put extra information on the cover
% page as needed:
% \ifCLASSOPTIONpeerreview
% \begin{center} \bfseries EDICS Category: 3-BBND \end{center}
% \fi
%
% For peerreview papers, this IEEEtran command inserts a page break and
% creates the second title. It will be ignored for other modes.
\IEEEpeerreviewmaketitle

\section{Introduction}
% no \IEEEPARstart
Given two (in)complete-specified boolean functions,
the Boolean Matching problem (under non-exact projective NPNP-equivalence) is to find the assignments to match PI and PO of one circuit to another and verify whether this assignment is appropriate with functional equivalence checking. The complexity of this problem is $O(2^{m+n}m!n!)$ (m and n is the PI/PO numbers of each circuit). Moreover, we have to consider the situation when the PI and PO numbers of the circuits do not match, resulting in an increase to the difficulty. We can directly map the problem to a QBF problem, and thus the complexity is in the extent of PSPACE-complete. To overcome the huge effort to solve PSPACE-complete problems, we have to search for ways to lower the complexity. In the most simplified problem - PP-equivalence, the problem can be solved more efficiently by observing the functional dependency, signature based SAT-solving and pruning [1,2,3]; in the latest researches [5,6], they used learning techniques to strengthen the algorithm and improve the speed. These are the keys to solve NP3 Boolean Matching problems with practical time and computing power.

% 給定兩個(in)completelt-specified Boolean function,Boolean Matching 的問題(under Non-exact Projective NPNP-equivalence)是給定PI與PI的對應關係與PO與PO的對應關係後,確認這兩部份的電路在function上是相同的。在問題的複雜度方面,是$O(2^{m+n}m!n!)$,並且需要考慮PI數量與PO數量不相同所會產生的問題,如果使用最直接的觀察,可以將整個問題轉換成一個QBF的問題,即屬於PSPACE-compltete的範圍。關於Boolean Matching 的最簡化版本(under PP-equivalence),在[1,2,3]中有提到,主要介紹了觀察functional dependency, signature based 並結合SAT-solver以及pruning的方式,而後面比較新的研究[5,6]則是利用learning的機制來加強演算的效率。

% An example of a floating figure using the graphicx package.
% Note that \label must occur AFTER (or within) \caption.
% For figures, \caption should occur after the \includegraphics.
% Note that IEEEtran v1.7 and later has special internal code that
% is designed to preserve the operation of \label within \caption
% even when the captionsoff option is in effect. However, because
% of issues like this, it may be the safest practice to put all your
% \label just after \caption rather than within \caption{}.
%
% Reminder: the "draftcls" or "draftclsnofoot", not "draft", class
% option should be used if it is desired that the figures are to be
% displayed while in draft mode.
%
%\begin{figure}[!t]
%\centering
%\includegraphics[width=2.5in]{myfigure}
% where an .eps filename suffix will be assumed under latex, 
% and a .pdf suffix will be assumed for pdflatex; or what has been declared
% via \DeclareGraphicsExtensions.
%\caption{Simulation results for the network.}
%\label{fig_sim}
%\end{figure}

% Note that the IEEE typically puts floats only at the top, even when this
% results in a large percentage of a column being occupied by floats.


% An example of a double column floating figure using two subfigures.
% (The subfig.sty package must be loaded for this to work.)
% The subfigure \label commands are set within each subfloat command,
% and the \label for the overall figure must come after \caption.
% \hfil is used as a separator to get equal spacing.
% Watch out that the combined width of all the subfigures on a 
% line do not exceed the text width or a line break will occur.
%
%\begin{figure*}[!t]
%\centering
%\subfloat[Case I]{\includegraphics[width=2.5in]{box}%
%\label{fig_first_case}}
%\hfil
%\subfloat[Case II]{\includegraphics[width=2.5in]{box}%
%\label{fig_second_case}}
%\caption{Simulation results for the network.}
%\label{fig_sim}
%\end{figure*}
%
% Note that often IEEE papers with subfigures do not employ subfigure
% captions (using the optional argument to \subfloat[]), but instead will
% reference/describe all of them (a), (b), etc., within the main caption.
% Be aware that for subfig.sty to generate the (a), (b), etc., subfigure
% labels, the optional argument to \subfloat must be present. If a
% subcaption is not desired, just leave its contents blank,
% e.g., \subfloat[].


% An example of a floating table. Note that, for IEEE style tables, the
% \caption command should come BEFORE the table and, given that table
% captions serve much like titles, are usually capitalized except for words
% such as a, an, and, as, at, but, by, for, in, nor, of, on, or, the, to
% and up, which are usually not capitalized unless they are the first or
% last word of the caption. Table text will default to \footnotesize as
% the IEEE normally uses this smaller font for tables.
% The \label must come after \caption as always.
%
%\begin{table}[!t]
%% increase table row spacing, adjust to taste
%\renewcommand{\arraystretch}{1.3}
% if using array.sty, it might be a good idea to tweak the value of
% \extrarowheight as needed to properly center the text within the cells
%\caption{An Example of a Table}
%\label{table_example}
%\centering
%% Some packages, such as MDW tools, offer better commands for making tables
%% than the plain LaTeX2e tabular which is used here.
%\begin{tabular}{|c||c|}
%\hline
%One & Two\\
%\hline
%Three & Four\\
%\hline
%\end{tabular}
%\end{table}


% Note that the IEEE does not put floats in the very first column
% - or typically anywhere on the first page for that matter. Also,
% in-text middle ("here") positioning is typically not used, but it
% is allowed and encouraged for Computer Society conferences (but
% not Computer Society journals). Most IEEE journals/conferences use
% top floats exclusively. 
% Note that, LaTeX2e, unlike IEEE journals/conferences, places
% footnotes above bottom floats. This can be corrected via the
% \fnbelowfloat command of the stfloats package.





\section{Preliminaries}
Given two circuits Cir1 and Cir2, the (PIs, POs) of each circuit are (X, F) and (Y, G) respectively. X, Y, F, G are sets of PIs/POs, and can presented as: $X = \{x_1, x_2, ... , x_n\}$, $Y = \{y_1, y_2, ... , y_m\}$, $F = \{f_1, f_2, ... , f_k\}$, $G = \{g_1, g_2, ... , x_l\}$. We use expression $MUX(A, c_1, c_2, ... ,c_t)$ to denote the output of a MUX, which uses the elements in set A to determine which of c\_1 to c\_t are assigned as the output.

% 給定兩個電路Cir1與Cir2,他們的PIs與POs分別為X,Y與F,G,代表PI或PO的集合,並且寫作$X = \{x_1, x_2, ... , x_n\}$, $Y = \{y_1, y_2, ... , y_m\}$, $F = \{f_1, f_2, ... , f_k\}$, $G = \{g_1, g_2, ... , x_l\}$,另外使用$MUX(A, c_1, c_2, ... ,c_t)$來表示以A集合中的變數來選取c_1至c_t 的其中一個,將其中一個值傳至輸出端。

\subsection{NP3 Boolean Matching}
Given two circuits Cir1 and Cir2, $\forall y_i \in Y$, $ y_i \in \{0, 1, x_1, \overline{x_1}, x_2, \overline{x_2}, ... , x_n, \overline{x_n}\}$. Under this circumstance, $\forall g_i \in G$, $g_i \in \{f_1, \overline{f_1}, f_2, \overline{f_2}, ... , f_n, \overline{f_n}, others\}$. The goal is to find an assignment that gives the maximum number of equivalent pair $(g_i \equiv f_j)$ under any possible input values.
% 給定兩個個電路Cir1與Cir2,對於電路Cir2的PI集合Y的元素,$y_i \in \{0, 1, x_1, \overline{x_1}, x_2, \overline{x_2}, ... , x_n, \overline{x_n}\}$,在此條件下,對於Cir2的PO集合G的元素,$g_i \in \{f_1, \overline{f_1}, f_2, \overline{f_2}, ... , f_n, \overline{f_n}, other\}$,目標是找到一組對應關係,使得給定任意的inputs,有最多對的$g_i \equiv f_j$。

\subsection{Quantified Boolean Formula}
Since we can map our problem into a QBF problem, we will now make a brief explanation for QBFs. A Quantified Boolean Formula queries whether formula $\phi$ is satisfiable under a series of quantified boolean variables with order. For example: 
% 我們可以將NP3 Boolean Matching 的問題轉成一個QBF的問題,而QBF是廣義版的Boolean satisfiability problem,以另一個說法,QBF詢問對於一連串Quantified Boolean varibles with order,formula \phi 是否有解,以下舉一個例子:

\[ \forall x\ \exists y\ \exists z\ .((x\lor z)\land y) \]

In general, we will write the formula in the following format:
% 在一般的狀況下,我們會將式子寫成這樣的形式:

\[ \exists X_1\ \forall X_2\ \cdots Q_nX_n .~\phi \]

Each $Q \in \{\forall, \exists\}$ represents a quantifier, and each $X_i$ represents a block consists of adjacent quantifier variables. Since many researches has already put forward efficient methods to solve QBF problems, mapping the original question to QBF problems can help reduce the complicated process for proving, but the downside is longer solving process.
% 每一個$Q \in \{\forall, \exist\}$代表一種 Quantifier,每一個$X_i$代表一個variables block,由相鄰的一樣 Quantifier 的變數所組成。解決QBF這樣形式的問題,近年來有非常多研究[6,7,8],因此如果將問題轉變成QBF,便能省去許多證明流程上的複雜性,但相對來說,會使得解決的效率變差。
    近年來比較新的研究有[2,3],而在ABC裡面所建構的QBF,只能解決2 level的QBF問題,它是利用隨機生成第一組$\forall$變數,並且呼叫 miniSAT[7],若SAT則檢查$\forall$部份是否為UNSAT,若UNSAT則將這一組assignment記錄下來,並尋找下一組,將前一組失敗的結果與新的AND起來達到pruning的效果。


\section{Implementation}
In this section we will explain how we construct this simple solver in detail. First we will talk about the tool we use - ABC. We will briefly introduce the internal data structure and the API we used. Then we will show the circuit structure in AIG. Finally, we will illustrate the usage of ILP in the problem, the method to iteratively optimize the solution, and how to construct circuits to represent in-equations such as $a+b+c \geq 2$.
% 在這個段落會詳細講解我們如何建構這個簡單的Solver,首先介紹我們所使用的ABC package,裡面內建的部份資料結構以及可以使用的API;接下來因為我們會使用AIG的方式來儲存整個電路,所以會展示電路的架構圖;最後一部分使用ILP來達到iterative遞增解決問題的方式,簡單說明如何以電路設計類似$a+b+c \geq 2$的不等式。
\subsection{ABC Package}
%<<<<<<< 920e56612cbbe2469913aaec0ed7326f74b9631d
ABC is A System for Sequential Synthesis and Verification written by Berkeley Logic Synthesis and Verification Group, it combines scalable logic optimization based on And-Inverter Graphs (AIGs), optimal-delay DAG-based technology mapping for look-up tables and standard cells, and innovative algorithms for sequential synthesis and verification.
% 我們主要使用read_verilog這個指令來有效的讀取兩個電路檔,並參考ABC中的miter與[9]來建構整個電路,
% ABC 是 A System for Sequential Synthesis and Verification 由 Berkely的Berkeley Logic Synthesis and Verification Group所建構,他可以用AIG的形式處理可擴張的電路,並且以DAG為基礎,最佳延遲為目標,在系統中運作時都會進行technology mapping。
% 我們主要使用read_verilog這個指令來有效的讀取兩個電路檔,並參考ABC中的miter與[9]來建構整個電路,
%=======
%ABC 是 A System for Sequential Synthesis and Verification 由 Berkely的Berkeley Logic Synthesis and Verification Group所建構,他可以用AIG的形式處理可擴張的電路,並且以DAG為基礎,最佳延遲為目標,在系統中運作時都會進行technology mapping。
%我們主要使用read\_verilog這個指令來有效的讀取兩個電路檔,並參考ABC中的miter與[9]來建構整個電路,
%>>>>>>> rebase
\subsection{AIG Structure}
An and-inverter graph (AIG) is a directed, acyclic graph  that represents a structural implementation of the logical functionality of a circuit or network.\\
每一個AIG中的node代表一個電路圖中的And gate,並且在每一個input端可以決定這個值是否要被翻轉。以這種形式表示電路來處理Boolean satisfiability在formal verification的領域裡有非常大的成效,包含model checking以及equivalence checking。
因此在將兩個電路讀進ABC system後,進行strash的動作,直接將整個電路圖轉變成為AIG的形式以進行接下來的操作,如圖1所示。
% 放ppt的圖
\subsection{ILP Circuit}
在圖中,上半部份是由n個MUX所組成,每一個MUX會從Cir2的outputs裡面選一個作為輸出,輸出之後與Cir1中的每一個output連接至一個XOR gate,將所有的XOR的輸出連接至一個OR gate後,將控制輸出是否為CONST0的MUX連接至ILP circuit,最後將ILP的輸出與OR gate的輸出and在一起,得到最後的OUTPUT。


\section{Experimental Results}
MUX on diffeent side of circuit

\section{Conclusion}
We use the advanced tool ABC as a base to do all of our implementations. First we read in two verilog files, construct the MUX groups and ILP circuits, and join the circuits to the network. After doing basic resync/strashing, we use the internal QBF function to derive the solution.
We overlooked many performance concerns in our current implementation: the number of MUXes are too large; the internal QBF solver is un-controllable, and we haven't resort to external superior QBF solvers.
So our solution is simple, complete, but rather inefficient.
% 我們使用ABC system中建構的AIG模組,讀取兩個verilog的電路檔,並連接MUX以及ILP電路,最後使用裡面內建的QBF solver來解決NP3 Boolean Matching的問題,在效能上有許多問題並沒有考量到,由於MUX個數過多,以及QBF solver的效能不太能控制,也沒有使用外部的solver,因此我們所建構的NP3 solver算是一個簡單完整的程式,只是並不能保證效能。

\section{Improvement}
To achieve better performance, it is impossible to throw the whole circuit into the QBF solver due to the massive search space. Putting this into consideration, we aim to improve in these few directions.
% 如果要使效能變快,一定不能整個電路都扔進QBF solver中,這樣的search space實在太過龐大,考量到這個部份,接下來會朝幾個方向來思考如何進步。

First is to observe the PO structural supports and functional supports for both circuits. We can use this information as the upper and lower bound. If the support range of the POs of the two circuits do not overlap, it means that the two POs are impossible to match. We can than construct a candidate matching list for each POs.
% 第一點是觀察兩個電路中每個PO的structural supports與functional supports數量,可以利用這兩個值作為上下界,並再擴張"兩個電路PI數量的差距",只要兩個PO的範圍沒有重疊則表示這兩個PO不可能functional equivalence,可以由此建構出每個Cir1的PO可能對應到Cir2的PO的表,並根據此表的對應關係來作MUX或是進一步的Combinational Equivalence Checking。

Second, by using the candidate list above, we can simulate the matching pair of POs to partition the PIs into groups. By the order of the partitioning process and the size of each groups, we can match the PI groups of the two groups, which drastically reduce the number of pairs.
% 再來則是對於兩個已知有機會相等的PO,如果先不考慮PI端的negation,可以利用[1]中所提到的方法,經由simulation來將Cir1與Cir2的PI分別作分堆的動作,並且依據分出堆的先後順序與大小排列,即可根據兩邊的分堆來作PI的對應,可以大量減少可能對應的對數。

Another approach is that after observing the support, we start from the POs with the smallest cones and use QBF solver to solve this partial circuit. If a solution exists, we use this information to assign PIs and recursively solve the rest of the circuit until a conflict occurs. If conflict occurs, we skip this group and continue the process starting from the next group.
% 另一個方式則是在做完supports的觀察後,決定先從PO的fanin con最小的開始作配對,直接扔進QBF solver,看是否有解,如果有解則將PI對應的順序記錄下來,以此為條件繼續作fanin con 數量第二小的PO的配對,持續直到conflict產生,可以跳過這組PO配對或者重頭來過,並以這樣的算法持續運作。

\section{Jod Division}
John Wu: Connect QBF solver, test automation, result parsing, output, presentation PowerPoint contents, report final touches and translation\\

Yen-Shi Wang: Construct all circuits in ABC, PowerPoint for presentation, report initial version \\

% conference papers do not normally have an appendix

\input{Improvement}
% use section* for acknowledgment
%\ifCLASSOPTIONcompsoc
  % The Computer Society usually uses the plural form
%  \section*{Acknowledgments}
%\else
  % regular IEEE prefers the singular form
%  \section*{Acknowledgment}
%\fi



% trigger a \newpage just before the given reference
% number - used to balance the columns on the last page
% adjust value as needed - may need to be readjusted if
% the document is modified later
%\IEEEtriggeratref{8}
% The "triggered" command can be changed if desired:
%\IEEEtriggercmd{\enlargethispage{-5in}}

% references section

% can use a bibliography generated by BibTeX as a .bbl file
% BibTeX documentation can be easily obtained at:
% http://mirror.ctan.org/biblio/bibtex/contrib/doc/
% The IEEEtran BibTeX style support page is at:
% http://www.michaelshell.org/tex/ieeetran/bibtex/
%\bibliographystyle{IEEEtran}
% argument is your BibTeX string definitions and bibliography database(s)
%\bibliography{IEEEabrv,../bib/paper}
%
% <OR> manually copy in the resultant .bbl file
% set second argument of \begin to the number of references
% (used to reserve space for the reference number labels box)
\begin{thebibliography}{1}

\bibitem{}
    Hadi Katebi and Igor L. Markov, Large-scale Boolean Matching, \emph{DATE}, 2010
\bibitem{}
    KLIEBER, William, et al. Solving QBF with free variables. In: \emph{International Conference on Principles and Practice of Constraint Programming.} Springer Berlin Heidelberg, 2013. p. 415-431.
\bibitem{}
    LONSING, Florian; BIERE, Armin. DepQBF: A dependency-aware QBF solver. Journal on Satisfiability, Boolean Modeling and Computation, 2010, 7: 71-76.
\bibitem{}
    C.F. Lai, J.-H.R. Jiang, and K.-H. Wang, BooM: A Decision Procedure for Boolean Matching with Abstraction and Dynamic Learning, \emph{DAC}, 2010
\bibitem{}    
    C.F. Lai, J.-H. R. Jiang, and K.-H. Wang, Boolean Matching of Function Vectors with Strengthened Learning, \emph{ICCAD}, 2010
\bibitem{}
    EÉN, Niklas; SORENSSON, Niklas. Translating pseudo-boolean constraints into SAT. \emph{Journal on Satisfiability}, Boolean Modeling and Computation, 2006, 2: 1-26.
\bibitem{}
    EÉN, Niklas; SÖRENSSON, Niklas. An extensible SAT-solver. In: \emph{International conference on theory and applications of satisfiability testing}. Springer Berlin Heidelberg, 2003. p. 502-518.

\end{thebibliography}




% that's all folks




\end{document}
